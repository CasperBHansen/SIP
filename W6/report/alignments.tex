%
% alignment.tex
%

\setcounter{section}{3}
\section{Affine and Projective Alignments}

\subsection{Homogeneous coordinate space}
To show why we prefer the homogeneous coordinate space over Euclidean
coordinate space, we consider a homogeneous point $p$ in $n$-nary space and a
series of transformations $M_0, M_1, \dots,M_m$ that we wish to apply to point
$p$ in that particular order.

Applying a transformation should be of the form $p_b = M_{i}p_a$, taking an
input vector $p_a$ producing an ouput vector $p_b$. This allows us to reason
that since $M_{i}a$ produces a vector, we can apply the exact same form to the
following transformation --- that is, $p_c = M_{i+1}p_b$. By induction it then
follows that $p_{final}(M_m(M_{m-1}(\dots(M_0)\dots)p_{original}$.

We can further argue that since $A(BC) = (AB)C$, we then have that the
concatenated transformations can be expressed by
$(M_m(M_{m-1}(\dots(M_0)\dots)p = (M_{m}M_{m-1} \dots M_{0})p$ and hence, we
can precalculate the entire sequence of transformations into one single
matrix, which can then be applied to any point, and thus eliviates many
redundant matrix multiplications. Since the translation transformation cannot
be carried out by matrix multiplication using Euclidean coordinate space, we
cannot apply the same optimization technique, and hence homogeneous coordinate
space is preferred.

\begin{multicols}{2}
    \subsection{Exact procrustes mapping}
    No, there is no such mapping because we're are limiting to the basic
    transformations; translation, scaling and rotation.
    \fig[scale=0.5]{4-2}{4-2}{Attempted procrustes mapping from $X$ to $Y$}
    Review the code to produce the figure above in appendix
    \ref{assign:procrustes-mapping}.

    \vfill\columnbreak

    \subsection{Exact affine mapping}
    Affine gives us only one more transformation to work with; namely shearing.
    This, however, is still not enough to make an exact mapping.
    \fig[scale=0.5]{4-3}{4-3}{Attempted procrustes mapping from $X$ to $Y$}
    Review the code to produce the figure above in appendix
    \ref{assign:affine-mapping}.
\end{multicols}

\newpage
\subsection{Projective transform}
\begin{multicols}{2}
    As far as I can see, we have all the parameters available that we need to
    make an exact mapping with projective, although the code I've written does
    not seem to work, and I have failed to find any bug in it (it can be
    reviewed in appendix \ref{assign:projective-transform}).

    The above figure to the right shows the output of the code. It looks like
    it produces a very squeezed rectangle.

    \vfill{\ }\columnbreak
    \fig[scale=0.5]{4-4}{4-4}
    {Faulty projective transformation mapping $X$ to $Y$}
\end{multicols}

