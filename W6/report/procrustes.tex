%
% procrustes.tex
%

\section{Procrustes Transformations}

\subsection{Free parameters and minimum $N$}
I see no extraordinary conditions in the described problem, and so I would
expect it to exhibit exactly the same free parameters as in the general case;
namely 4 free parameters; $\alpha$, $\gamma$, $\lambda_1$ and $\lambda_2$,
with which we can express the entire transformation matrix by
\begin{align}
    \label{eqn:procrustes}
    T &=
    \begin{bmatrix}
        \alpha & \gamma & \lambda_1 \\
        -\gamma & \alpha & \lambda_2 \\
        0 & 0 & 1
    \end{bmatrix}
\end{align}

which decomposes into three matrices (translation $X$, rotation $R$ and
scaling $S$);
\begin{align}
    X =
    \begin{bmatrix}
        1 & 0 & \beta_x \\
        0 & 1 & \beta_y \\
        0 & 0 & 1
    \end{bmatrix}
    \qquad
    R =
    \begin{bmatrix}
        \cos \theta & \sin \theta & 0 \\
        -\sin \theta & \cos \theta & 0 \\
        0 & 0 & 1
    \end{bmatrix}
    \qquad
    S =
    \begin{bmatrix}
        s & 0 & 0 \\
        0 & s & 0 \\
        0 & 0 & 1
    \end{bmatrix}
\end{align}

where

\begin{align}
    \alpha = s \cos \theta \quad
    \gamma = s \sin \theta \quad
    \lambda_1 = s(\beta_x \cos \theta + \beta_y \sin \theta) \quad
    \lambda_2 = s(-\beta_x \sin \theta + \beta_y \cos \theta)
\end{align}

Because the minimization of the procrustes translation vector $\vec{t}$ is
merely a negated average of the input vectors there is no dependence on $N$
for this parameter. Likewise, so is the case of scaling; if we let $N=1$, then
$s$ becomes $1$, which produces an identity matrix, not transforming anything.
I failed to find a mathematical argument as to the required $N$ in order to
produce the rotation matrix, can however state that I would imagine that since
we must build the two eigenvectors $U$ and $V$, we should need at least $N=2$.

\subsection{Finding procrustes parameters}
In this assignment I've commented out the interactive code, which asks the
user to pick out four sequential points for each of the images presented.
These have been made static for a particular set of points that I've picked
out myself --- a rendition of the result is shown below. The code to produce
this can be reviewed in appendix \ref{assign:finding-procrustes}.

\fig{2-2}{2-2}{Rendition of the pre-selected point outcome}

From the output of running \code{procrustes} on the input and target points I
found that the scale $s$ is $1.0977$, whilst the translation vector $\vec{t}$
was $(-88.3279, 10.6171)^T$ and the rotation matrix was given by

\begin{align*}
    \begin{bmatrix}
        0.9820 & -0.1888 \\
        0.1888 &  0.9820
    \end{bmatrix}
\end{align*}

From these we can built the corresponding homogeneous matrices for
translation $X$, rotation $R$ and scaling $S$, and computing the product
transformation matrix $T = SRX$ we can then read out the procrustes parameters
one by one.

\begin{align}
    \footnotesize
    \begin{bmatrix}
        1.0977 & 0 & 0 \\
        0 & 1.0977 & 0 \\
        0 & 0 & 1
    \end{bmatrix}
    \begin{bmatrix}
        0.9820 & -0.1888 & 0 \\
        0.1888 & 0.9820 & 0 \\
        0 & 0 & 1
    \end{bmatrix}
    \begin{bmatrix}
        1 & 0 & -88.3279 \\
        0 & 1 & 10.6171 \\
        0 & 0 & 1
    \end{bmatrix}
    &=
    \footnotesize
    \begin{bmatrix}
        1.0780 & -0.2073 & -76.9113 \\
        0.2073 & 1.780 & -113.5232 \\
        0 & 0 & 1
    \end{bmatrix}
\end{align}

This gives us the full transformation matrix, and by equation
(\ref{eqn:procrustes}) we can now read out the parameters from this matrix.
\begin{align}
    \alpha = 1.0780 \qquad
    \gamma = -0.2073 \qquad
    \lambda_1 = -76.9113 \qquad
    \lambda_2 = -113.5232
\end{align}

