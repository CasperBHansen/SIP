%
% translation.tex
%

\section{Experiments on Translation}

\subsection{Translation matrix filter}

\begin{multicols}{2}
    
    The 2-dimensional homogeneous translation matrix is given by
    \begin{align}
        \begin{bmatrix}
            1 & 0 & t_x \\
            0 & 1 & t_y \\
            0 & 0 & 1
        \end{bmatrix}
        \text{, if we let $w=1$.}
    \end{align}
    
    For a translation matrix one pixel to the right, we let $t_x=1$ and
    $t_y=0$. And the corresponding filter kernel matrix would then be given by
    \begin{align}
        \begin{bmatrix}
            0 & 0 & 0 \\
            1 & 0 & 0 \\
            0 & 0 & 0
        \end{bmatrix}
    \end{align}
    
    \vfill\columnbreak
    
    It is easy to express $\widetilde{I}(x,y)$ by $I(x,y)$; each pixel in the
    transformed image $\widetilde{I}$ corresponds to the pixel immediately to
    the left of it of $I$. That is

    \begin{align}
        \widetilde{I}(x,y) = I(x-1,y)
    \end{align}
    
    An example of applying this simple 1-pixel translation kernel is given in
    appendix \ref{assign:translation}.

\end{multicols}

\subsection{Generalized translation}
The code for this assignment is given in appendix
\ref{assign:general-translation}, and the generalized translation function is
given in appendix \ref{func:translate}. An example rendition of its effects is
shown below in figure \ref{fig:1-2}.

\fig{1-2}{1-2}{Example translation}

I've chosen to use a wrapping boundary condition. I found this to be the most
desirable boundary condition for translation, although I could have simply
ignored pixels outside the boundaries of the image.

\newpage
\begin{multicols}{2}
    \subsection{Fourier transform translation}
    The code used to produce figure \ref{fig:1-3} below can be review in
    appendix \ref{assign:fourier-translation}.
    \fig[scale=0.30]{1-3}{1-3}{Fourier translation using $\vec{t} = (2,1)^T$}
    As I couldn't do this using a more direct approach I sought some help on
    the internet, and so this solution code is largely inspired by an
    answer\footnote{http://www.stackoverflow.com/questions/25827916/matlab-shifting-an-image-using-fft} on StackOverflow.
    
    \vfill{\ }\columnbreak
    
    \subsection{Non-integer fourier translation}
    The code used to produce figure \ref{fig:1-4} below can be review in
    appendix \ref{assign:non-integer-fourier-translation}.
    \fig[scale=0.5]{1-4}{1-4}
    {Fourier translation using $\vec{t} = (1.5,0.5)^T$}
    
    As can be seen from the above figure, non-integer translations produces
    artifacts in the image, as apparent of the square image, and causes
    inversions as well, as evident of the \code{peak} produced image.
\end{multicols}

