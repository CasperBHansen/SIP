%
% theory.tex
%

\section{Theory}
% In this section we address the theoretical questions of Fourier transforms.

\subsection{Fourier Series and Transform Differences}
% what is the difference between a fourier series and the Fourier transform?
% pp. 115
The Fourier series produce an approximation of a function by way of a weighted
sum of complex exponentials (or harmonic functions), the accuracy of which
increases with the number of terms employed. The Fourier transform is a
synthesis of a Fourier series (summation) of discrete frequencies into a
function of continuous (integral) frequencies.

\subsection{Continuous Fourier Transforms}
% prove that the continuous Fourier transform of a real and even function is
% real and even
Consider the complex exponential term of the Fourier transform $e^{-ik_xx}$.
Because the integral is done over the interval $[-\infty;\infty]$, for any
intermediate discrete value $a$ of the exponent we have that;
\begin{align}
    e^{ia}
    &= e^0 (\cos a + i \sin a)
    = \cos a + i \sin a
    \\
    e^{-ia}
    &=e^0 (\cos a - i \sin a)
    = \cos a - i \sin a
\end{align}
From this general example, we see that the imaginary part of the equations
cancel out for each symmetric term over the integral, making it real.

% prove that even and oddness is preserved; maybe a simple example x^2 can be
% approximated by one term

\subsection{Derivation of Fourier Transform}
% Derive the continuous Fourier transform of \delta (x - d) + \delta (x + d)
% for some constant d
Regardless of the constant $d$ the integral over the Dirac (or impulse-)
function will stay at exactly $1$. So, each of the terms must become $1$, and
so the Fourier transform for entire expression $\delta(x - d) + \delta(x + d)$
must then be $2$. 
% pp. 123, dirac (impulse function) is always 1, both becomes 1 for integrals
% going from -infinity to +infinity, therefore it must yield the result 2

\subsection{The Box Function}
Consider the box function
\begin{align}
    b_a(x) =
    \begin{cases}
        \frac{1}{a}     & \mbox{iff. $|x| \leq \frac{a}{2}$} \\
        0               & \mbox{otherwise}
    \end{cases}
\end{align}
Show that

\subsubsection{\mdseries $\int_{-\infty}^{\infty} b_a(x) dx = 1$}
For the general case, we can say that the integral will always have a length
over the $x$-axis of $2 \frac{a}{2}$ and the height of it will always be
$\frac{1}{a}$. If we consider the formula for the area of a box and apply
these findings we have that $A = \frac{1}{a} \cdot 2 \frac{a}{2}$. Reducing
this, we find that the area is always equal to 1.
\begin{align}
    \frac{1}{a} \cdot 2 \frac{a}{2}
    = \frac{1}{a} \cdot \frac{2a}{2}
    = \frac{1}{a} \cdot a
    = \frac{a}{a}
    = 1
\end{align}

\subsubsection{\mdseries The continuous Fourier transform of $b$ using (5.10)
is $B(k) = \frac{1}{ak\pi} \sin \frac{ak}{2}$. Rewrite $B(k)$ using the
$\sinc(x) = \frac{\sin x}{x}$ function.}
\begin{align}
    \frac{1}{ak\pi} \sin \left( \frac{ak}{2} \right)
    = \frac{1}{\pi} \frac{1}{ak} \sin \left( ak \frac{1}{2} \right)
    = \frac{1}{2\pi} \frac{1}{ak \frac{1}{2} } \sin \left( ak \frac{1}{2} \right)
    = \frac{1}{2\pi} \frac{ \sin \left( ak \frac{1}{2} \right) }{ak \frac{1}{2} }
    = \frac{1}{2\pi} \sinc \left( \frac{ak}{2} \right)
\end{align}

\subsubsection{\mdseries $\lim_{a \rightarrow 0} B(k) = \frac{1}{2\pi}$ (Hint:
$\lim_{x \rightarrow 0} \frac{\sin x}{x} = 1$). Does this prove an entry in
Table 5.2?}
Since $\lim_{x \rightarrow 0} \frac{\sin x}{x} = 1$, then letting $a
\rightarrow 0$ for $\frac{1}{2\pi} \sinc \left( \frac{ak}{2} \right)$ we have
that $\lim_{a \rightarrow 0} \frac{1}{2\pi} \sinc \left( \frac{ak}{2} \right)
= \frac{1}{2\pi} \cdot 1$, which yields $\lim_{a \rightarrow 0} B(k) =
\frac{1}{2\pi}$.

% If we let $W = \frac{1}{2\pi}$ and 

\subsubsection{\mdseries The filter $b$ has compact support in space (only
uses a small set of neighbouring pixels in $x$). Is the same true in the
frequency domain, $k$? Explain your answer.}
Because of the inherent revertibility of Fourier transforms, such information
of $b$ must carry over from spatial domain into the frequency domain, so as
to make the frequency domain revertible. Therefore, this must be true of the
frequency domain as well.
