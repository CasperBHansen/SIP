%
% sinogram.tex
%

\section{Sinogram}
In medical imaging we do not have an image source per se. The way the input is
produced is by shooting, or projecting, rays through an object, and then read
the scattered rays as they exit the object. So, all we have is a numerical
value for each projection that is read. If we think of these as the integral
over the projected rays, our task is then to recover the integrated function.

We can express this idea by the Radon transform, given in equation
(\ref{eqn:radon-transform}).
\begin{align}
    \label{eqn:radon-transform}
    p(\xi,\phi) &= \int f(x,y) \delta(x \cos \phi + y \sin \phi - \xi)
    {\ }dx{\ }dy
\end{align}

Where $f(x,y)$ is the function we wish to recover, $\delta$ is the Dirac delta
(or impulse) function, $\phi$ is the angle of projection, and $\xi$ is the
offset of the projected line from the origin. Figuratively, at least in our
case, we are trying to express, mathematically, all line projections (or
exiting rays) covering the image plane for all angles $\phi \in M$.

In MatLab, we can simplify the calculation of this mathematical expression by
simply rotating the source image, using \code{imrotate}, and integrate
(\code{sum}) for each angle $\phi$ (see \ref{functions:sinogram} for
implementation).

\fig{sino-box}{1-a}{0.80}{Sinograms of \file{box.png} and \file{mir.tif} using
180 projections}

Figure (\ref{fig:sino-box}) above shows a rendition of the corresponding
sinogram for the \file{box.png} and \file{mir.tif} images, calculated using
the algorithm discussed, and the MatLab program given in
\ref{assignment:sinogram}. As can be seen, the image produced has $M$ columns,
each of which represent the integral for a given $\phi$.
