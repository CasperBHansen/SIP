%
% backprojection.tex
%

\newpage
\section{Backprojection}
The reconstruction of the original image, or in the case of medical imaging,
the actual image we wish to create, is called backprojection. It is expressed
mathematically, as below

\begin{align}
    \label{eqn:backprojection}
    f_{\text{BP}}(x,y)
    &= \int_{0}^{\pi} p(x \cos \phi + y \sin \phi, \phi)
    {\ }d\phi
\end{align}

As we can see, we are integrating over a hemisphere ($\phi = [0;\pi]$), just
as with the projections in producing the sinogram, we mustn't let the angle
$\phi$ surpass $180^\circ$ --- just in radians.

% assignment 2
\subsection{Reconstruction}
By integrating over $p(\xi,\phi)$ in the hemisphere interval we are
essentially estimating a reconstruction of the original image from the
sinogram, as the ones shown below in figure \ref{fig:bp-sinograms} that we
will reconstruct (see figure \ref{fig:bp-reconstructed}).

\fig{bp-sinograms}{2-a}{0.75}{Sinograms to be reconstructed}

The implementation of this algorithm can be reviewed in
\ref{functions:backprojection}. This algorithm uses a precalculated set of
coordinates using \code{meshgrid} (line 5) in conjunction with a mid-point
offset (line 4) to find the indices (lines 10--11) over which we want to
integrate (line 17).

\fig{bp-reconstructed}{2-b}{0.75}{Reconstructed images from sinograms}

% assignment 3
\subsection{Filtered}
The algorithm for filtered backprojection remains largely the same as that of
ordinary backprojection, although it involves convoluting with a ramp filter
for each summation. This has been built into the previously mentioned MatLab
implementation of backprojection (see \ref{functions:backprojection}) using an
optional filter argument. This makes it extensible, and the ramp filter is
simply supplied as an argument in \ref{assignment:backprojection}.

\fig{fbp-reconstructed}{3-a}{0.80}{Reconstructed images from sinograms using
filtered backprojection}

As we can see by comparing figure \ref{fig:bp-reconstructed} with the results
of the filtered backprojection, shown above in figure
\ref{fig:fbp-reconstructed}, there is a clear improvement in distinguishing
the features of the image, as opposed to the latter attempt at reconstructing
the images. 

% assignment 4
\subsection{Projections}
As we increase the number of projections $M$ used to produce the sinograms, we
see a steady increase in clarity of the reconstructed image from its sinogram.
In figure \ref{fig:projections} below I've chosen to show renditions for very
small values of $M$, as this allows us to observe the individual projection
lines very clearly.

\fig{projections}{4}{0.70}{Sinograms and corresponding reconstructions as $M$
increases}

It should be noted that for $M > N$ I stop seeing any effect within the
reconstructed image.
