%
% invfilter.tex
%

\section{Inverse Filtering}

% 4
\subsection{Linear Shift Invariant}
\begin{multicols}{2}
\noindent
    Figure (\ref{fig:lsi-degraded}) shows the result of the program code
    given in \ref{appendix:lsi-degraded}.
    \vfill{\ }\columnbreak
    \fig{lsi-degraded}{4}{0.5}{The linear shift invariantly (LSI) degraded
    result}
\end{multicols}

% 5
\subsection{Inverse and Band-pass Filtering}
Considering the inverse filter $\frac{1}{H(k_x,k_y)}$ we see that for the
resulting equation, when applying it to $G(k_x,k_y) + N(k_x,k_y)$, where
$N(x,y)$ is the additive noise, we have that for instances of $G(k_x,k_y) <
N(k_x,k_y)$ the noise term becomes dominant, an example of which is given in
figure (\ref{fig:inv-bandpass}) (left) where we allow most of the frequencies
to pass right through. Since noise often consists of high-frequency signals,
the high-pass (see appendix \ref{appendix:highpass}) filter allows for
filtering out some of these signals, notably at the expense of $G(k_x,k_y)$ as
well.

\fig{inv-bandpass}{5}{0.8}{Inverse band-pass filter at different cut-off
frequencies}

From the tests I have run on this approach I find that the result is generally
not good.

\subsection{Wiener Filter}
The Wiener filter attempts to overcome the limitations of the straight-forward
---and naive--- approach taken by simply filtering out at a fixed cut-off
frequency, as is the case with the high-pass filter. It does so by weighting
the contributions of $G(k_x,k_y)$ and $N(k_x,k_y)$, respectively, by way of
the signal/noise power signal.

\fig{inv-wiener}{6}{0.75}{Inverse Wiener results}

The results of using this filter is by far an improvement over the band-pass
filter method.
