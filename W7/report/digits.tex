%
% digits.tex
%

\section{Digit Recognition}
The strategy I followed to relax the method was to produce a very thin,
minimal base shape of each digit to project hits from and a very forgiving,
bold shape of each digit to project misses from. Each of these are shown in
figure \ref{fig:5a} below.

The first (thin) shape was produced using the skeletonization technique after
a slight erosion which makes for simpler skeletons, and the latter by simply
eroding the inverse.

\fig{5a}{5a}{Original digits (top), skeletonized hit structural elements
(middle) and expanded (or eroded) miss structural elements}

These were then used as the structural elements for detecting the digit shapes
in the source image, where $B_1$ is designated as being the skeletonized shape
and $B_2$ as the bold shape.

\fig{5b}{5b}{Points of recognition, based on hit-or-miss calculation}

As evident of figure \ref{fig:5b} above, we see that we do indeed detect all
three digits in the source image. Review the code that was used to produce
these results in appendix \ref{assign:recognition}.

