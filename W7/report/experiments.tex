%
% experiments.tex
%

\section{Experimentation}
\begin{multicols}{2}
    \noindent
    In each application of the three operations I've used the same three
    structural elements, the parameters of which are given in the table.
    
    \begin{align}
        \text{TopHat}_B(A) &= A - ((A \ominus B) \oplus B) \label{eqn:tophat} \\
        \text{BotHat}_B(A) &= ((A \oplus B) \ominus B) - A \label{eqn:bothat}
    \end{align}
    
    \vfill\columnbreak

    \begin{figure}[H]
        \center
        \begin{tabular}{|l|c|c|}
            \hline {\bf Name} & {\bf Shape} & {\bf Parameters} \\ \hline
            line & {\tt 'line'} & (50,1) \\ \hline
            disk & {\tt 'disk'} & 3 \\ \hline
            square & {\tt 'square'} & 6 \\ \hline
        \end{tabular}
        \caption{Structural elements used}
        \label{fig:experimental-se}
    \end{figure}
\end{multicols}

\newpage
\subsection{Hit-and-Miss}
On the left of figure \ref{fig:3a} below, we see that the line does indeed
find the lines as expected, in the middle we get anything which fits the disc
shape of radius 3, and lastly so is the case for the larger square, which
finds less objects, but still some.

\fig{3a}{3a}{Hit-and-miss experiments}

\subsection{Top hat}
The top hat operation (\ref{eqn:tophat}) subtracts the detected objects of the
opening from the original image, the effects of which can be seen in figure
\ref{fig:3b} below.
\fig{3b}{3b}{Top hat experiments}

\subsection{Bottom hat}
The bottom hat operation (\ref{eqn:bothat}) subtracts the original image from
the closed, the effects of which can be seen in figure \ref{fig:3c} below.
\fig{3c}{3c}{Bottom hat experiments}

