%
% sketch.tex
%

\section{Sketch Outcomes}
By applying the {\it open} ($\circ$) and {\it close} ($\bullet$) operations,
given in equation (\ref{eqn:openclose}) below, to the provided image data, we
get two smaller objects for the {\it opened} image, and a larger one for the
{\it closed} one.

\begin{align}
    A \circ B = (A \ominus B) \oplus B
    \qquad
    A \bullet B = (A \oplus B) \ominus B
    \label{eqn:openclose}
\end{align}

What is happening is that the {\it open} operation is first eroding away outer
details, followed by a dilation which closes the gaps left behind. For the
{\it closing} operation we first dilate the image, causing an expansion of the
object, followed by an erosion which then strips away excess outer detail.

\fig{2}{2}{Original (left), opened (middle) and closed (right)}

Had we used a diamond structural element of radius $2$ we would get a single
object in both cases. This is due to the fact that the diamond would be able
to reach across the inner gaps of the original objects and that the little
blob in the lower right area is too far away for it to reach across.
