%
% background.tex
%

\section{Background Normalization}
After some experimentation I found that in and around the radii interval 7--9
the opened image begins to settle into a stabile image. As we can see in
figure \ref{fig:4a}, when the radius is set to 7 we stil have a slight
artifact on the left, but with radii 8 and 9 there is virtually no immediate
visible change in the produced image.

\fig{4a}{4a}{Background extractions of \file{rice.png} in the interval 7--9.}

We choose a radius of 8 because it is the first in the interval to produce an
even background result and the last from which notable changes can be seen,
and if we were to continue increasing the radius we would end up with the
original, or much like it --- which would defeat the purpose.

\fig{4b}{4b}{Optimal background normalization of \file{rice.png}}

As apparent from figure \ref{fig:4b} the background appears to have vanished
from the image when we subtract it from the original image.
